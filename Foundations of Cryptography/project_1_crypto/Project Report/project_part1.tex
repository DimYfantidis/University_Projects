\documentclass[12pt]{article}

\usepackage[english, greek]{babel}
\usepackage{csquotes}
\newcommand{\lt}{\latintext}
\newcommand{\gt}{\greektext}

\usepackage{mathtools}
\usepackage{amsmath}
\usepackage{amsfonts}
\usepackage{cancel}
\usepackage[dvipsnames]{xcolor}
\usepackage{listings}
\usepackage{hyperref}

\newcommand{\xor}{\ensuremath{\oplus}}
\newcommand{\nul}{\ensuremath{\hphantom{*}}}
\newcommand{\red}[1]{\ensuremath{\color{red} #1}}
\newcommand{\green}[1]{\ensuremath{\color{green} #1}}
\newcommand{\blue}[1]{\ensuremath{\color{blue} #1}}
\newcommand{\magenta}[1]{\ensuremath{\color{magenta} #1}}
\newcommand{\violet}[1]{\ensuremath{\color{violet} #1}}
\newcommand{\brown}[1]{\ensuremath{\color{brown} #1}}
\newcommand{\teal}[1]{\ensuremath{\color{teal} #1}}

\usepackage[margin=1in]{geometry}

\title{\textbf{{\lt Project I} \\ Συμμετρική Κρυπτογραφία}}
\author{Υφαντίδης Δημήτριος (ΑΕΜ: 3938)}
\date{\today}

\begin{document}
\maketitle

\section*{Εισαγωγή}

Το παρόν αποτελεί την αναφορά του πρώτου μέρους της εργασίας 
στο μάθημα {\lt ``}Θεμελιώσεις Κρυπτογραφίας{\lt "}. \\

\noindent
Η αναφορά είναι γραμμένη σε {\lt LATEX} και μεταγλωττίστηκε 
από τον {\lt \textbf{MiKTeX} compiler (ver: One MiKTeX Utility 
1.6 - MiKTeX 23.1)}. \\

\noindent
Οι υλοποιήσεις των ασκήσεων έγιναν σε γλώσσα 
\textbf{\lt Python (v 3.10.6)}. Για κάθε άσκηση που 
ζητάει υλοποίηση σε κώδικα, υπάρχει ο αντίστοιχος φάκελος:
\begin{center}
{\lt proj\_1\_crypto\_3938}/κώδικας/{\lt ex*}
\end{center}
Όπου βρίσκεται ένα μοναδικό {\lt python script, 
\textbf{ex*.py}} που περιέχει την υλοποίηση και ενδεχομένως 
κάποια {\lt ``.txt"} αρχεία ή άλλους πόρους για το πρόγραμμα. \\

\vspace{0.2cm}
\noindent
Σχεδόν όλοι οι μαθηματικοί συμβολισμοί που εμφανίζονται 
στο κείμενο είναι κοινώς αποδεκτοί. Εξαίρεση αποτελούν 
οι ακόλουθοι, που διευκρινίζονται:
\begin{itemize}
	\item $\mathbf{x\:mod\:y:}$ Το υπόλοιπο της ακέραιας διαίρεσης του $x$ με το $y$ (όχι κλάση ισοδυναμίας).
	\item $\mathbf{\mathbb{N}}:$ Το σύνολο των φυσικών αριθμών \textbf{συμπεριλαμβάνοντας} το 0, δηλ. $[0,\:1,\:2,\:...]$.
	\item $\mathbf{\mathbb{N}^{*}}:$ Το σύνολο των φυσικών αριθμών \textbf{χωρίς} το 0, δηλ. $[1,\:2,\:3,\:...]$.
\end{itemize}

\pagebreak

\section*{Άσκηση 1 (2.2)}
Έχουμε:
\begin{itemize}
	\item $g(x) = x^{2} + 3x + 1,\; x \in R$
	\item $f(x) = x^{5} + 3x^{3} + 7x^{2} + 3x^{4} + 5x + 4,\; x \in R$
	\item $x_{0}$ κάποια ρίζα της $g\;\;\Rightarrow\;\; g(x_{0}) = 0$
\end{itemize}
Υπολογίζουμε την $f(x_{0})$:

\begin{align*}
	f(x_{0}) &= x_{0}^{5} + {\color{blue} 3x_{0}^{3}} + 7x_{0}^{2} + 
	{\color{red}3x_{0}^{4}} + 5x_{0} + 4 \\
	&= x_{0}^{5} + {\color{red}3x_{0}^{4}} + {\color{blue} 3x_{0}^{3}} + 7x_{0}^{2} + 5x_{0} + 4 \\
	&= x_{0}^{5} + 3x_{0}^{4} + {\color{blue} x_{0}^{3}} + {\color{blue} 2x_{0}^{3}} 
	+ {\color{ForestGreen} 7x_{0}^{2}} + {\color{DarkOrchid} 5x_{0}} + {\color{Mahogany}4} \\
	&= x_{0}^{5} + 3x_{0}^{4} + x_{0}^{3} + 2x_{0}^{3} +
	{\color{ForestGreen} 6x_{0}^{2}} + {\color{DarkOrchid} 2x_{0}} + 
	{\color{ForestGreen} x_{0}^{2}} + {\color{DarkOrchid} 3x_{0}} + {\color{Mahogany}1} + 
	{\color{Mahogany}3} \\
	&= (x_{0}^{5} + 3x_{0}^{4} + x_{0}^{3}) + (2x_{0}^{3} +
	6x_{0}^{2} + 2x_{0}) + (x_{0}^{2} + 3x_{0} + 1) + 3 \\
	&= x_{0}^{3}(x_{0}^{2} + 3x_{0} + 1) + 2x_{0}(x_{0}^{2} +
	3x_{0} + 1) + (x_{0}^{2} + 3x_{0} + 1) + 3 \\
	&= x_{0}^{3}\cdot \cancel{g(x_{0})} + 2x_{0}\cdot \cancel{g(x_{0})} + \cancel{g(x_{0})} + 3
	\Rightarrow \\
	f(x_{0}) &= 3
\end{align*}
\noindent
Άρα, αν $\mathcal{E} = \{E, D, \mathcal{K}, \mathcal{M}, \mathcal{C}\}$ το κρυπτοσύστημά μας τότε:
\begin{itemize}
	\item $\mathcal{M} = \mathcal{C} = \{\alpha:1, \beta:2, ..., \omega\}$
	\item $\mathcal{K} = \{f(x_{0})\} = \{3\}$
	\item $E(m, k)|_{k=3}\; =\; (m + k - 1)\:mod\:24 + 1\:|_{k=3}\; = \;(m + 2)\:mod\:24 + 1 = c$
	\item $D(c, k)|_{k=3}\; =\; (m - k - 1)\:mod\:24 + 1\:|_{k=3}\; = \;(m - 4)\:mod\:24 + 1 = m$
\end{itemize}
Είναι δηλαδή ένα σύστημα μετατόπισης, όπου η αρίθμηση δεν ξεκινάει από το $0$, αλλά από το $1$
(από εκεί προκύπτουν και οι επιπλέον άσσοι στις συναρτήσεις $E$ και $D$).
Προκύπτει ότι, αφού εισάγουμε το κρυπτομήνυμα \textbf{{\lt ``}οκηθμφδζθγοθχυκχσφθμφμχγ{\lt "}}, 
δίνεται ως έξοδος το αποκρυπτογραφημένο μήνυμα \textbf{{\lt ``}μηδεισαγεωμετρητοσεισιτω{\lt "}}.\\ 

\vspace{0.1in}

\noindent
(Υλοποίηση: \textbf{κώδικας/{\lt ex1/ex1.py}})

\pagebreak

\section*{Άσκηση 2 (2.3)}

Χρησιμοποιώντας τη μέθοδο του {\lt Friedman} προκύπτει ότι το μήκος κλειδιών και των δύο μηνυμάτων είναι 7. Μια {\lt brute force} υλοποίηση εύρεσης κλειδιού βρίσκεται μέσα στο {\lt script ex2.py} όπου ψάχνει όλες τις δυνατές συμβολοσειρές μήκους 7 με τους 26 λατινικούς χαρακτήρες ($26^7$ συνδυασμοί). \\

\noindent
Εκτυπώνονται μόνο οι αποκρυπτογραφήσεις των οποίων το κείμενο είναι σχετικά αναγνώσιμο, δηλαδή η συχνότητα εμφάνισης των γραμμάτων είναι κοντά με την πραγματική.

\vspace{0.3in}

\noindent
(Υλοποίηση: \textbf{κώδικας/{\lt ex1/ex1.py}})

\pagebreak

\section*{Άσκηση 3 (2.4)}

Έχουμε, $\:E: \{0, 1\}^{16} \rightarrow \{0, 1\}^{16}$ με 
$\:E(m) = m\:\xor\:(m << 6)\:\xor\:(m << 10)\: = c$. Γράφουμε τις δυαδικές ακολουθίες με τη μορφή διανυσμάτων:

\begin{align*}
\begin{bmatrix}
c_{15} \\
c_{14} \\
c_{13} \\
c_{12} \\
c_{11} \\
c_{10} \\
c_{9} \\
c_{8} \\
c_{7} \\
c_{6} \\
c_{5} \\
c_{4} \\
c_{3} \\
c_{2} \\
c_{1} \\
c_{0} \\
\end{bmatrix} &=
\begin{bmatrix}
m_{15} \\
m_{14} \\
m_{13} \\
m_{12} \\
m_{11} \\
m_{10} \\
m_{9} \\
m_{8} \\
m_{7} \\
m_{6} \\
m_{5} \\
m_{4} \\
m_{3} \\
m_{2} \\
m_{1} \\
m_{0} \\ 
\end{bmatrix} \xor
\begin{bmatrix}
m_{9} \\
m_{8} \\
m_{7} \\
m_{6} \\
m_{5} \\
m_{4} \\
m_{3} \\
m_{2} \\
m_{1} \\
m_{0} \\
m_{15} \\
m_{14} \\
m_{13} \\
m_{12} \\
m_{11} \\
m_{10} \\
\end{bmatrix} \xor
\begin{bmatrix}
m_{5} \\
m_{4} \\
m_{3} \\
m_{2} \\
m_{1} \\
m_{0} \\ 
m_{15} \\
m_{14} \\
m_{13} \\
m_{12} \\
m_{11} \\
m_{10} \\
m_{9} \\
m_{8} \\
m_{7} \\
m_{6} \\
\end{bmatrix} = 
\begin{bmatrix}
m_{15} \xor m_{9\nul} \xor m_{5\nul} \\
m_{14} \xor m_{8\nul} \xor m_{4\nul} \\
m_{13} \xor m_{7\nul} \xor m_{3\nul} \\
m_{12} \xor m_{6\nul} \xor m_{2\nul} \\
m_{11} \xor m_{5\nul} \xor m_{1\nul} \\
m_{10} \xor m_{4\nul} \xor m_{0\nul} \\
m_{9\nul} \xor m_{3\nul} \xor m_{15} \\
m_{8\nul} \xor m_{2\nul} \xor m_{14} \\
m_{7\nul} \xor m_{1\nul} \xor m_{13} \\
m_{6\nul} \xor m_{0\nul} \xor m_{12} \\
m_{5\nul} \xor m_{15} \xor m_{11} \\
m_{4\nul} \xor m_{14} \xor m_{10} \\
m_{3\nul} \xor m_{13} \xor m_{9\nul} \\
m_{2\nul} \xor m_{12} \xor m_{8\nul} \\
m_{1\nul} \xor m_{11} \xor m_{7\nul} \\
m_{0\nul} \xor m_{10} \xor m_{6\nul} \\
\end{bmatrix}
\end{align*} \\
Μπορούμε οπτικά να συμπεράνουμε ότι
{\large
\[
c_{i}\;=\;m_{i}\:\xor\:m_{(i+10)\,mod\,16}\:\xor\:m_{(i+6)\,mod\,16},\;\; i=0, 1, ..., 15
\]
}
ή αλλιώς 
{\large
\[
c_{i\,mod\,16}\;=\;m_{i\,mod\,16}\:\xor\:m_{(i+10)\,mod\,16}\:\xor\:m_{(i+6)\,mod\,16},\;\; \forall i \in \mathbb{N}
\]
}

\vspace{0.1in}
\noindent
Προκύπτει (μετά από πειραματισμούς) ότι αν κάνουμε 
\textbf{{\lt XOR}} κατά μέλη τα $c_{2}$, $c_{4}$, $c_{6}$, $c_{8}$, 
$c_{10}$ παίρνουμε το έκτο ψηφίο του του αρχικού μηνύματος ($m_{6}$). Παρατίθεται η απόδειξη:
{\large
\begin{center}
\begin{tabular}{cccccccc}
       & $c_{(i+2)\,mod\,16}$ & = & $\red{m_{(i+2)\,mod\,16}}$& $\xor$ &$\violet{m_{(i+12)\,mod\,16}}$& $\xor$ & $\blue{m_{(i+8)\,mod\,16}}$\\
       & $c_{(i+4)\,mod\,16}$ & = & $\green{m_{(i+4)\,mod\,16}}$& $\xor$ &$\brown{m_{(i+14)\,mod\,16}}$& $\xor$ & $\magenta{m_{(i+10)\,mod\,16}}$\\
       & $c_{(i+6)\,mod\,16}$ & = & $\mathbf{m_{(i+6)\,mod\,16}}$& $\xor$ &$\teal{m_{(i+\cancel{16})\,mod\,16}}$& $\xor$ & $\violet{m_{(i+12)\,mod\,16}}$\\
       & $c_{(i+8)\,mod\,16}$ & = & $\blue{m_{(i+8)\,mod\,16}}$& $\xor$ &$\red{m_{(i+2)\,mod\,16}}$& $\xor$ & $\brown{m_{(i+14)\,mod\,16}}$\\
$\xor$ & $c_{(i+10)\,mod\,16}$ & = & $\magenta{m_{(i+10)\,mod\,16}}$& $\xor$ &$\green{m_{(i+4)\,mod\,16}}$& $\xor$ & $\teal{m_{(i+\cancel{16})\,mod\,16}}$\\

\hline
 & & & & & & & $m_{(i+6)\,mod\,16}$ \\
\end{tabular}
\end{center}
}
Δηλαδή, 

{\large
\[
c_{(i+2)\,mod\,16} \xor c_{(i+4)\,mod\,16} \xor c_{(i+6)\,mod\,16} \xor c_{(i+8)\,mod\,16} \xor c_{(i+10)\,mod\,16} = m_{(i+6)\,mod\,16}
\]
}

\pagebreak

Θέτοντας $u = i - 10 \Rightarrow i = u + 10$ η παραπάνω σχέση γράφεται ως:
{\large
\[
m_{u\,mod\,16} = c_{(u+12)\,mod\,16} \xor c_{(u+14)\,mod\,16} \xor c_{u\,mod\,16} \xor c_{(u+2)\,mod\,16} \xor c_{(u+4)\,mod\,16}
\]
}
\indent
Όπως είδαμε στην αρχή, η αριστερή κυκλική κύλιση κατά 6 αντιστοιχεί στη μαθηματική έκφραση $\mathbf{m_{(i+10)\,mod\,16}}$ ενώ η αριστερή κυκλική κύλιση κατά 10 αντιστοιχεί στη μαθηματική έκφραση $\mathbf{m_{(i+6)\,mod\,16}}$.\\
\indent
Γενικότερα η κυκλική κύλιση κατά $\alpha,\; 0 < \alpha < N$
σε έναν αριθμό των $N$ {\lt bits} μετατοπίζει τα πρώτα $\alpha$ {\lt MSBs} στη θέση των πρώτων $\alpha$ {\lt LSBs}. 
Άρα το {\lt LSB} (θέση 0) τώρα θα είναι το ψηφίο που πρηγουμένως ήταν στη θέση $(N-1) - \alpha$. 
Άρα έκφραση που πήραμε ως αποτέλεσμα για τα ψηφία του $m$ ως συνάρτηση των ψηφίων του $c$ μπορεί να μετασχηματιστεί στην έκφραση:

{\large
\[
m\; = \;(c << 4)\:\xor\:(c << 2)\:\xor\:c\:\xor\:(c << 14)\:\xor\:(c << 12)\; = \;D(c)
\]
}

\vspace{0.1in}

\noindent
Το παραπάνω συμπέρασμα επιβεβαιώνεται πειραματικά από το 
{\lt script ``ex3.py"} καθώς τερματίζει με 
\textbf{{\lt error\_flag = False}}. 
\vspace{0.2in}

\noindent
(Υλοποίηση: \textbf{κώδικας/{\lt ex3/ex3.py}})


\section*{Άσκηση 4 (2.5)}

Ένα σύστημα μετατόπισης θα μπορούσε να έχει τέλεια ασφάλεια αν το κλειδί άλλαζε για κάθε χαρακτήρα με τυχαίο, ομοιόμορφο τρόπο. \\ 
Συγκεκριμένα, έστω το σύστημα 
$\mathcal{E} = \{E, D, \mathcal{K}, \mathcal{M}, \mathcal{C}\}$ 
με
\begin{itemize}
	\item $\mathcal{K} = \mathcal{M} = \mathcal{C} = \{0, 1, ..., 23\}$
	\item $E(m, k) = (m + k)\:mod\:2 = c$
	\item $D(c, k) = (c - k)\:mod\:2 = m$
\end{itemize}
Έστω ένα κείμενο $t$ μήκους $\ell$ χαρακτήρων. Αν $m_{0},\:m_{1} \in \mathcal{M}$ δύο οποιοιδήποτε χαρακτήρες του κειμένου $t\in\mathcal{M}^{\ell}$ θα ισχύει η εξίσωση τέλειας 
ασφάλειας:
\begin{center}
	$Pr(k \xleftarrow{\$}\mathcal{K}: E(k, m_{0}) = c)\;=\; 
	Pr(k \xleftarrow{\$}\mathcal{K}: E(k, m_{1}) = c)$
\end{center}
αφού κάθε κρυπτογραφημένος χαρακτήρας $c$ μπορεί να είναι το αποτέλεσμα 
μετατόπισης κατά οποιοδήποτε αριθμό $k$ θέσεων.  \\
\indent
Συμπερασματικά προκύπτει ότι κάθε συγκεκριμένο γράμμα δε θα 
μετατρέπεται σε ένα άλλο συγκεκριμένο γράμμα, άλλα σε οποιοδήποτε άλλο 
γράμμα για κάθε ξεχωριστή του εμφάνιση (άρα και δεν υπάρχει κίνδυνος 
όσον αφορά την εύρεση του κλειδιού με στατιστικές μεθόδους κλπ).\\ 
\indent
Αυτή η υλοποίηση ενός συστήματος μετατόπισης μοιάζει με το {\lt One Time Pad}.

\pagebreak

\section*{Άσκηση 5 (2.6)}

\indent
Ο πίνακας για την μετατροπή των συμβόλων σε δυαδικές ακολουθίες 
των 5 {\lt bit} υλοποιείται ως ένα λεξικό (συγκεκριμένα, δύο λεξικά για τη μετατροπή από σύμβολο σε {\lt bits} και από {\lt bits} σε σύμβολο). \\
Η συνάρτηση κρυπτογράφησης είναι ίδια με τη συνάρτηση αποκρυπτογράφησης 
\[
D(x, y) = E(x, y) = x \xor y
\]
όπως είναι φανερό και στον κώδικα. \\
Το κείμενο που θα χρησιμοποιηθεί ως μήνυμα βρίσκεται στο αρχείο 
\textbf{{\lt ``sample\_text.txt"}}. \\
\indent
Να σημειωθεί ότι το κλειδί επιλέγεται ψευδοτυχαία με τη χρήση 
της βιβλιοθήκης {\lt random}. Στην πραγματικότητα στο {\lt OTP} 
το κλειδί τυχαίο, όμως επειδή το κλειδί πρέπει να έχει το ίδιο 
μήκος με το κείμενο, η επιλογή ενός κλειδιού με το χέρι είναι 
υπερβολή στην συγκεκριμένη περίπτωση.\\
\indent
Το πρόγραμμα εμφανίζει (με τη σειρά) το τυχαίο κλειδί, το κρυπτογραφημένο κείμενο και το αποκρυπτογραφημένο κείμενο, που σε κάθε περίπτωση είναι ίδιο με το περιεχόμενο του αρχείου.

\vspace{0.2in}

\noindent
(Υλοποίηση: \textbf{κώδικας/{\lt ex5/ex5.py}})

\section*{Άσκηση 6 (3.6)}
Η άσκηση υλοποιήθηκε σε {\lt Python}.
\begin{itemize}
\item Συνάρτηση \textbf{{\lt is\_prime(n: int)}}, επιστρέφει σε χρόνο $\mathcal{O}(\sqrt{n})$ αν ο $n$ είναι 
πρώτος.
\item Συνάρτηση \textbf{{\lt prime\_factors(n: int)}}, επιστρέφει μια λίστα [$p_1,\:p_2,\:...,\:p_k$] όπου 
\[
\prod_{i=1}^{k}p_i\; = \;n, \;\;\; p_i\; \text{πρώτοι αριθμοί}
\]
\item Συνάρτηση \textbf{{\lt divisors(n: int)}}, επιστρέφει 
λίστα με όλους τους θετικούς διαιρέτες του $n$.
\item Συνάρτηση \textbf{{\lt mu(n: int)}}, υλοποίηση της συνάρτησης {\lt Möbius}, $\mu(n)$.
\item Συνάρτηση \textbf{{\lt N\_2(n: int)}}, το ζητούμενο της 
άσκησης.
\end{itemize}
Το πρόγραμμα διαλέγει 10 τυχαίους ακεραίους, $\mathbf{n_i}$, 
στο διάστημα $[1,\:100]$. Εκτυπώνει, με την ακόλουθη 
σειρά, τους διαιρέτες του, τους πρώτους παράγοντές του 
και το πλήθος των ανάγωγων πολυωνύμων βαθμού $n_i$ 
στο σώμα $\mathbb{F}_2$. Τα ακόλουθα αποτελέσματα 
εκτυπώνονται για $n = 30$: \\
\nul \\
{ \lt \texttt{
[30] info: \\
 -- divisors: [1, 2, 3, 5, 6, 10, 15, 30] \\
 -- prime factorization: [2, 3, 5] \\
 -- N\_2(30) = 35790267 \\
}}

\vspace{0.1in}

\noindent
(Υλοποίηση: \textbf{κώδικας/{\lt ex6/ex6.py}})

\pagebreak

\section*{Άσκηση 7 (3.8)}
\indent
Όπως και σε προηγούμενες ασκήσεις, το {\lt script} χρηιμοποιεί 
λεξικά για τις κωδικοποιήσεις και αποκωδικοποιήσεις των 
32 δοσμένων χαρακτήρων σε δυαδικές ακολουθίες των 
{\lt 5-bit} (όπως επιβάλλει ο πίνακας της άσκησης 3.5) 
και, φυσικά, την υλοποίηση της {\lt XOR}.\\
\indent
Ο αλγόριθμος {\lt RC4} απαιτεί {\lt 8-bit} κωδικοποίηση 
των χαρακτήρων, όμως οι χαρακτήρες μας κωδικοποιούνται σε 
{\lt 5 bits} ο καθένας. Μια πρώτη σκέψη θα ήταν να κάνουμε
{\lt padding} τρία μηδενικά μπροστά από τις δυαδικές 
αναπαραστάσεις των χαρακτήρων. Αυτό όμως δεν αποτελεί 
λύση του προβλήματος καθώς αν \texttt{\lt S[i] > 31} για κάποιο 
$i$, τότε το αποτέλεσμα  της {\lt XOR} ενός χαρακτήρα με το 
\texttt{\lt S[i]} θα προκαλούσε υπερχίλειση. \\
\indent
Έτσι υλοποιήθηκε, απλά, ο {\lt RC4} με όριο το 32 αντί για 
το 256. Άρα ο $S$ είναι $S = [0, 1, ..., 31]$ αντί για 
$S = [0, 1, ..., 255]$, οι δείκτες $i,\:j$ μηδενίζονται κάθε 
32 επαναλήψεις αντί για 256 κλπ. \\
\indent
Για την κρυπτογράφηση και την αποκρυπτογράφηση χρησιμοποιείται 
η ίδια συνάρτηση \texttt{{\lt rc4()}} που υλοποιεί τον 
αλγόριθμο αρχικοποίησης, την κατσκευή μετάθεσης και τον
αλγόριθμο κρυπτοροής ταυτόχρονα.\\ 
\indent
Για μήνυμα 
\texttt{\lt "MISTAKESAREASSERIOUSASTHERESULTSTHEYCAUSE"} και 
κλειδί \texttt{\lt "HOUSE"} παίρνουμε ως κρυπτοκείμενο το 
\texttt{\lt "IGD!APO-TJUQPDSPMAOZUIAZ(VF(VFGQ.IIWMB(WX"}. 
Βάζοντας το κρυπτοκείμενο στην ίδια συνάρτηση και το ίδιο 
κλειδί παίρνουμε ξανά το αρχικό μήνυμα, πράγμα που 
επιβεβαιώνει ότι οι συναρτήσεις κρυπτογράφησης και 
αποκρυπτογράφησης είναι ίδιες. \\


\vspace{0.1in}

\noindent
(Υλοποίηση: \textbf{κώδικας/{\lt ex7/ex7.py}})

\section*{Άσκηση 8 (4.3)}
Το {\lt script} υλοποιεί τον τύπο υπολογισμού της διαφορικής 
ομοιομορφίας:
{\large
\[Diff(S) \;=\; \max_{x \in F_{2}^{n}\:-\:\{0\},\; y \in F_{2}^{m}}
|\{z \in F_{2}^{n}:\; S(x \oplus z) \oplus S(z) \;=\; y\}|
\]
}

\noindent
Για $n=6,\: m = 4$ και $S:\;\{0, 1\}^6 \rightarrow \{0, 1\}^4$ 
με τύπο του {\lt S-box (4.2.3)}, το πρόγραμμα τερματίζει 
εμφανίζοντας ως έξοδο \texttt{\lt Diff(S) = 14}.\\

\noindent
Μερικές πληροφορίες για τον κώδικα:
\begin{itemize}
	\item Λίστα \texttt{\lt S\_i:} υλοποίηση του εσωτερικού 
	{\lt lookup table} της συνάρτησης \texttt{\lt S}
	\item Συνάρτηση  \texttt{\lt S(x: str) -> str:} η 
	συνάρτηση $S:\;\{0, 1\}^6 \rightarrow \{0, 1\}^4$
	\item Συνάρτηση  \texttt{\lt xor(x: str, y: str) -> str:} 
	επιστρέφει $x \xor y$
	\item Συνάρτηση 
	\texttt{\lt differential\_uniformity(...):} 
	δέχεται τη συνάρτηση \texttt{\lt S} και το μήκος 
	των συμβολοσειρών εισόδων της. Επιστρέφει το $Diff(S)$
\end{itemize}

\vspace{0.1in}
\noindent
(Υλοποίηση: \textbf{κώδικας/{\lt ex8/ex8.py}}, αποτελεί αντιγραφή του κώδικα για το δεύτερο ερώτημα του 
{\lt homework 4} με μικρο-αλλαγές)

\pagebreak

\section*{Άσκηση 9 (4.7)}

Η άσκηση 4.7 προϋποθέτει την ύπαρξη υλοποίησης του 
κρυπταλγορίθμου {\lt AES}. Δεδομένου ότι δεν υπάρχει κάποιο 
προεγκατεστημένο πακέτο της {\lt Python} με την εν λόγω 
υλοποίηση, χρησιμοποιήθηκε το πακέτο \textbf{\lt 
PyCryptodome (v 3.17)} - 
(\href{https://pycryptodome.readthedocs.io/en/latest/}
{\lt \color{blue} documentation link}). 
Η εγκατάσταση έγινε στη γραμμή εντολών με την εντολή: 
\texttt{\lt pip install pycryptodome} (Αντίστοιχα, 
πληροφορίες για το περιβάλλον {\lt conda} 
παρέχονται \href{https://anaconda.org/conda-forge/pycryptodome}
{\color{blue} εδώ}). \\
\indent
Όλα τα εργαλεία της βιβλιοθήκης βρίσκονται στο {\lt module 
\textbf{Crypto}}. Αναφέρται στο {\lt documentation} ότι η 
{\lt PyCryptodome} παρουσιάζει προβλήματα αν είναι ταυτόχρονα 
εγκατεστημένη η {\lt PyCrypto} (παλιά {\lt PyCryptodome}). 
Σε αυτή την περίπτωση πρέπει να εγκατασταθεί η 
{\lt PyCryptodomeX} με την εντολή \texttt{\lt 
pip install pycryptodomex}. Στην περίπτωση εγκατάστασης αυτής 
της ανεξάρτητης εκδοχής τα εργαλεία βρίσκονται στο 
{\lt module \textbf{Cryptodome}}. \\
\indent
Σχετικά με τον {\lt AES}, υπάρχουν 
\href{https://pycryptodome.readthedocs.io/en/latest/src/cipher/aes.html}{\color{blue}οδηγίες στο {\lt documentation}} που 
χρησιμοποιήθηκαν για την υλοποίηση της άσκησης (καθώς και από 
μια άλλη πηγή που θα αναφερθεί αργότερα). \\
\indent
Το αρχείο {\lt ``messages.txt"} περιέχει 78 μηνύματα που θα 
χρησιμοποιηθούν για το στατιστικό τεστ που θα διαπιστώσει το 
βαθμό διάχυσης του αλγορίθμου {\lt AES}.
Παρατίθενται τα βήματα του προγράμματος:
\begin{enumerate}
\item Διαβάζονται τα μηνύματα από το αρχείο και αποθηκεύονται 
σε έναν πίνακα.
\item Αποθηκεύονται με την ίδια σειρά τα μηνύματα οι 
επεξεργασμένες εκδοχές τους (ένα {\lt bit-flip}) σε άλλον 
πίνακα. 
\item Δημιουργείται ένα κλειδί των 16 {\lt bytes} (Άρα 
{\lt AES-128}).
\item Δημιουργούμε δύο {\lt AES} αντικείμενα με το 
κλειδί, ένα σε {\lt ECB} λειτουργία και ένα σε {\lt CBC} 
λειτουργία.
\item Δημιουργείται μια δομή ({\lt zip object}) με ζεύγη 
$(c_i,\:c_i')$ που προκύπτουν από την κρυπτογράφηση των 
$(m_i,\:m_i')$ με {\lt AES-128 ECB mode}, 
όπου $m_i,\:m_i'$ μηνύματα των 256 {\lt bit} 
που διαφέρουν κατά ένα {\lt bit}. Κάνουμε το ίδιο και 
για το {\lt CBC mode}. Να σημειωθεί ότι τα μηνύματα είναι 
λιγότερα από 32 {\lt bytes} οπότε γίνεται {\lt padding} 
με τη χρήση του {\lt \texttt{Padding.pad}} 
(Για αυτό το θέμα χρησιμοποιήθηκε κώδικας από 
\href{https://stackoverflow.com/questions/52181245/valueerror-data-must-be-aligned-to-block-boundary-in-ecb-mode}{\color{blue}αυτό} το 
{\lt post} στο {\lt Stack Overflow}).
\item Δημιουργείται ένα αρχείο στο οποίο καταγράφεται το 
ποσοστό διαφορετικών {\lt bits} μεταξύ $c_i$ και $c_i'$ 
(κάνοντας {\lt XOR} και μετρώντας to πλήθος των άσσων του 
αποτελέσματος) για κάθε $i\:=\:1, 2, ..., 78$. Επίσης 
καταγράφεται και ο μέσος όρος αλλαγής ψηφίων τόσο στο 
αρχείο, όσο και στην οθόνη. Αυτή η διαδικασία γίνεται δύο 
φορές, μία για κάθε {\lt AES} λειτουργία.
\end{enumerate}
Παρατηρείται ότι το ο {\lt AES-128} δε χαρακτηρίζεται από 
υψηλή διάχυση στη λειτουργία {\lt ECB} καθώς 
το μέσο ποσοστό ψηφίων που επηρεάστηκαν στο κρυπτομήνυμα 
είναι περίπου $25\%$, δηλαδή $<50\%$. \\
\indent
Κατά τ' άλλα αν εκτελεστεί ο κρυπταλγόριθμος στη {\lt CBC} 
λειτουργία τότε το αποτέλεσμα αλλάζει δραστικά, με περίπου 
$50.5\%-51\%$ αλλαγή ψηφιων. Επομένως το φαινόμενο της 
χιονοστιβάδας παρατηρείται στον αλγόριθμο {\lt Rijndael}, 
όχι όμως σε όλες τις λειτουργίες του. \\

\vspace{0.1in}

\noindent
(Υλοποίηση: \textbf{κώδικας/{\lt ex9/ex9.py}})

\pagebreak

\section*{Άσκηση 10}

\begin{enumerate}
	\item Εισάγουμε το {\lt \textbf{course-1-introduction.pdf}} σε \href{https://products.groupdocs.app/metadata/export/pdf}{\color{blue} αυτό} το {\lt PDF Metadata Viewer}.
	\item Αντιγράφουμε το {\lt Base64 string} που βρίσκεται στη θέση {\lt XMP $\rightarrow$ TIFF:ARTIST}
	\vspace{-0.2in}
	\begin{center}
\textbf{\footnotesize \lt 	aHR0cHM6Ly9jcnlwdG9sb2d5LmNzZC5hdXRoLmdyOjgwODAvaG9tZS9wdWIvMTUv}
	\end{center}
	και το εισάγουμε σε έναν \href{https://www.base64decode.org/}{\color{blue} αποκωδικοποιητή}. Έτσι, προκύπτει ο σύνδεσμος:
	\begin{center}
	{\lt \color{blue} \url{https://cryptology.csd.auth.gr:8080/home/pub/15/}}
	\end{center}
	\item Η ιστοσελίδα γράφει \textquote{O χημικός 
	{\lt Walter White} βρήκε μέσα στο εργαστήριο του το 
	εξής μήνυμα: {\lt \#2-75-22-6!}}. Το δοσμένο {\lt string} 
	δεν είναι ο κωδικός, όμως, αν γίνει κάποια συσχέτιση με τη χημεία, συμπεραίνουμε ότι οι αριθμοί αυτοί είναι ατομικοί αριθμοί στοιχείων, άρα $(2, 75, 22, 6) \rightarrow (He, Re, Ti, C)$. Επομένως, εισάγοντας στο {\lt secure.zip} τον κωδικό {\lt \textbf{\#heretic!}}, τότε αυτό ανοίγει.
	\item Μέσα στο {\lt secret.zip} βρίσκεται το {\lt secret.txt}. Αποκωδικοποιώντας το μεγάλο {\lt Base64 string} προκύπτει μια φωτογραφία του {\lt Bobby Fischer} να παίζει σκάκι. Στα μεταδεδομένα της φωτογραφίας υπάρχει ένα \href{https://tinyurl.com/26ru4359}{\lt \color{blue} tinyurl}.
	\item Η ιστοσελίδα γράφει (σε αλγεβραϊκή μορφή) την τρέχουσα κατάσταση μιας παρτίδας σκάκι. Κάνοντας {\lt prompt} το \href{https://openai.com/blog/chatgpt}{\lt \color{blue}ChatGPT} με εντολή:
	{\lt
	\begin{verbatim}
	Calculate the next best move:
1.e4 e5 2.f4 exf4 3.Bc4 g5 4.Nf3 g4 5.O-O gxf3 6.Qxf3 Qf6 7.e5
Qxe5 8.d3 Bh6 9.Nc3 Ne7 10.Bd2 Nbc6 11.Rae1 Qf5 12.Nd5 Kd8
13.Bc3 Rg8 14.Bf6 Bg5 15.Bxg5 Qxg5 16.Nxf4 Ne5 17.Qe4 d6 18.h4
Qg4 19.Bxf7 Rf8 20.Bh5 Qg7 21.d4 N5c6 22.c3 a5 23.Ne6+ Bxe6
24.Rxf8+ Qxf8 25.Qxe6 Ra6 26.Rf1 Qg7 27.Bg4 Nb8
	\end{verbatim}
	}
αυτό απαντάει:
	{\lt
	\begin{verbatim}
In this position, White has a strong initiative, but Black is still defending well. One possible continuation for White is:
28.Rf7
...
	\end{verbatim}
	}
	\item Υπολογίζουμε το {\lt \texttt{hashlib.md5(b'Rf7').hexdigest()}} που είναι ίσο με: 
	
	\textbf{{\lt f1f5e44313a1b684f1f7e8eddec4fcb0}} που είναι όντως το κλειδί του {\lt secure2.zip}. Το {\lt mastersecret.txt} περιέχει ως απάντηση το {\lt hash:}
	
	\textbf{{\lt be121740bf988b2225a313fa1f107ca1}}
\end{enumerate}



\end{document}